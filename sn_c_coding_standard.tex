\documentclass[article, 10pt]{memoir}
\usepackage[utf8]{inputenc}
\usepackage[T1]{fontenc}
\usepackage[danish]{babel}
\usepackage{listings}
\usepackage{tikz}
\usepackage{multicol}
\usepackage{pdflscape}

\usepackage{cmbright}
\usepackage[scaled=0.85]{DejaVuSansMono}

\let\tempone\itemize
\let\temptwo\enditemize
\renewenvironment{itemize}{\tempone\firmlist}{\temptwo}

\let\tempthree\enumerate
\let\tempfour\endenumerate
\renewenvironment{enumerate}{\tempthree\firmlist}{\tempfour}

\renewcommand{\topfraction}{0.85}
\renewcommand{\textfraction}{0.1}
\renewcommand{\floatpagefraction}{0.85}

\setlength{\parindent}{2pt}
\setlength{\parskip}{6pt}

\setlrmarginsandblock{1.7cm}{1.7cm}{*}
\setulmarginsandblock{2.5cm}{2.5cm}{*}
\setlength{\evensidemargin}{\oddsidemargin}
\checkandfixthelayout

\lstset{
  breaklines=true, 
  basicstyle=\ttfamily, 
  frame=none,
  numbers=none, 
  literate={ø}{{\o}}1 {æ}{{\ae}}1{å}{{\aa}}1,
  showtabs=false,
  showspaces=false,
  showstringspaces=false,
  captionpos=b,
  keywordstyle=\color{blue!70!black},
  commentstyle=\color{red!50!black}
}

\title{C kodestandard}
\author{Søren Nørgaard}
\date{Sidst opdateret: \today}
\begin{document}
\maketitle
\begin{multicols}{2}

    {
        \itshape
        Dette dokumenterer en kodestil, der med fordel kan anvendes i sproget C. Hvis ikke andre kodestile er krævet, kan denne med fordel følges, for at skabe ens udseende kode i projektet. At anvende ens kodestil i et projekt gør koden lettere at læse, da alting er at finde på samme plads.

        Denne kodestil tager udgangspunkt i Kerninghan \& Richie's kodestil (K\&R), og har desuden hentet inspiration fra Linux- kernelen mm.

        På de sidste sider er et kodeeksempel, hvor de fleste dele er eksemplificeret. (Koden gør absolut intet, men gør det måske lettere at forstå hvad der menes).

        God kodning!
    }
    \tableofcontents

    \chapter{Visuel organisering}
    Visuel opsætning af kode, herunder indentering og parentes-sætning (curly brackets).

    \section{Indentering}
    \begin{itemize}
    \item Identering er 8 spaces.
    \item Labels, ifm. \texttt{goto}'s, indenteres ikke, og sættes helt til venstre.
    \item Labels/\texttt{case}s ifm. \texttt{switch}e indenteres på samme niveau som \texttt{switch}.
    \end{itemize}

    \section{Curly brackets}
    \begin{itemize}
    \item Curly brackets sættes på samme linje som udtrykket det tilhører (\texttt{if}, \texttt{while} etc.), pånær ved funktioner!
    \item Ved funktioner sættes curly brackets på linjen under funktionsudtrykket, som en linje for sig.
    \item Brug ikke curly brackets efter \texttt{if}, \texttt{else}, \texttt{for}, \texttt{while}, hvis kun en enkelt statement følger. Denne sættes på en linje for sig, og indenteres.
    \item I \texttt{if .. else if .. else}-strukturer, sættes \texttt{else if }og \texttt{else} på samme linje som tidligere niveaus afsluttende curly bracket.
    \item Efter sluttende curly bracket i en \texttt{do-while} konstruktion, sættes \texttt{while} på samme linje.
    \end{itemize}

    \section{Whitespace}
    \begin{itemize}
    \item Whitespace benyttes konsekvent til at opdele dele af koden, der ikke hører sammen -- ligesom man ville gøre med afsnit i en bog.
    \item Der benyttes (som hovedregel) ikke mere end 1 tom linje til at opdele kode.
    \end{itemize}

    \section{Linjelængde}
    \begin{itemize}
    \item Linjer skal holdes under 80 karakterer. Dette er gammel praksis, men er meget smart hvis man vil printe kode i en rapport eksempeltvis.
    \item Hvis funktioners parametre fylder mere end 80 karakteres, kan linjen knækkes ved et komma, og alignes under funktionens begyndende parentes (under første parameter).
    \item Hvis andre strukturer \texttt{if}, \texttt{for} osv. bliver for lange, kan disse flyttes ned på næste linje, og indentes ligesom funktioner ovenfor. Deles der ved operatorer kan disse med fordel placeres på den nye linje.
    \item Lange konstante strings, må gerne overstige 80 karakterer, for at gøre det lettere at søge på dem.
    \end{itemize}

    \section{Editor-opsætning}
    Meget at ovenstående kan gøres automatisk af din editor, hvis den sættes ordentligt op. Herunder er en vejledning til at sætte et par editorer op.

    \subsection{Vim}
    I Vim gives ovennævnte effekt ved at placere følgende i opstartsfilen \verb|~/.vimrc|:
    \begin{lstlisting}
set encoding=utf-8
set tabstop=4
set shiftwidth=4
set expandtab
set autoindent
set cinoptions=:0,l1,t0,g0,(0
syntax enable
    \end{lstlisting}

    \subsection{Emacs}
    I Emacs indsættes følgende i opstartsfilen \texttt{~/.emacs} eller \verb|~/.emacs.d/init.el|:
    \begin{lstlisting}
(setq-default tab-width 4)
(setq-default c-basic-offset 4)
(setq-default indent-tabs-mode nil)

(prefer-coding-system       'utf-8)
(set-default-coding-systems 'utf-8)
(set-terminal-coding-system 'utf-8)
(set-keyboard-coding-system 'utf-8)
(setq default-buffer-file-coding-system 'utf-8)
(setq x-select-request-type '(UTF8_STRING COMPOUND_TEXT TEXT STRING))
    \end{lstlisting}

    \chapter{Navngivning}
    \begin{itemize}
    \item Alle navne skrives på engelsk.
    \item Globale funktioner (der ikke er \texttt{static}), bør have sigende, unikke navne som \texttt{spi\_send\_byte()}.
    \item Funktioner der er static (kun kan ses i den kildefil de er i), bør have korte navne, der er sigende indenfor filens eget område (eks. \texttt{startclk()})
    \item Samme gælder variable: Globale variable skal have lange/sigende navne, mens lokale variable bør holdes korte, og sigende indenfor sin givne funktion.
    \item Funktions- og variabelnavne skrives kun med små bogstaver, og ord er separerede med \texttt{\_}.
    \item Ofte brugte variabelnavne er:
        \begin{itemize}
        \item \texttt{i, j, k} til indeksering.
        \item \texttt{p, q} til pointere.
        \item \texttt{c} til karakterer.
        \item \texttt{s} til string.
        \item \texttt{x, y, z} til floats/doubles.
        \item \texttt{l} til long.
        \item \texttt{n} til ints.
        \end{itemize}
    \item Ved ``modsatte'' funktioner benyttes modsat navngivning. Eksempeltvis: \texttt{setclk()} og \texttt{getsck()}, \texttt{readbyte()} og \texttt{writebyte()} osv.
    \item Defines skrives med \texttt{BLOKBOGSTAVER}, og makroer ligeså. Eks. \texttt{\#define MAX(A,B) a>b?a:b}. 

        Hvis makroer opfører sig som funktioner, skrives de som funktioner. Eks. \texttt{\#define\ udelay(us) \_delay\_us(us)}.
    \item Fejlkoder defineres som negative tal, er sigende for koden den beskriver, og indeholder et \texttt{E} for error. Eks: \texttt{\#define SPI\_ENACK} for en fejl ved et spi-modul, der bliver NACK’et.
    \end{itemize}

    \chapter{Kommentarer og kodedokumentation}

    \section{Dokumenterende kommentarer}
    \begin{itemize}
    \item Før hver globale funktion, indsættes en blokkommentar, der starter med \texttt{/**}, og har en \texttt{*} indenteret med 1 space for hver linje. Blokken afsluttes med en tom linje med \texttt{*/}.
    \item Blokken startes med en kort beskrivelse af hvad funktionen gør.
    \item For hver parameter indsættes en linje begyndende med \texttt{@param x Beskrivelse af x.}.
    \item For returværdier indsættes en linje som: \texttt{@return 0 = success, SPI\_ENACK = No acknowledge.}, hvor alle returværdiers betydning dokumenteres.
    \item Dokumenterende kommentarer til funktioner forefindes ved funktionernes prototyper (i \texttt{.h}-filer).
    \item static funktioner (i \texttt{.c}-filer) kan med fordel indeholde en kommentar over funktionen, der forklarer returværdier, men bør ellers være sigende i sig selv.
    \item Dokumentation skrives på engelsk.
    \end{itemize}

    \section{Almindelige kommentarer}
    \begin{itemize}
    \item Forsøg at lave din kode så forståelig som muligt uden kommentarer, og brug dem kun hvor det er nødvendigt.
    \item Skriv ikke unødvendige kommentarer (``\texttt{i++; /* i is incremented by 1. */}'').
    \item Kommenter så vidt muligt for enden af en linje, så koden kan læses uforstyrret. Hvis en stor kommentar skal knyttes, skriv den da som en blokkommentar som her:
        \begin{lstlisting}
/* 
 * FIXME:
 * Some work needs to be done, to let
 * the user know an ACK has occured. 
 */
        \end{lstlisting}
        med en tom linje under og over, så kommentaren står for sig selv.
    \item Kommentarer skrives på engelsk, og skrives som om man skriver til kode. Eks: \texttt{/* Find the student with the best grades. */}.
    \end{itemize}

    \chapter{Filopdeling}
    Skrives eksempeltvis et main-program, filen \texttt{main.c}, med et bibliotek bestående af \texttt{\{spi.h, spi.c\}}, opdeles filerne som følger. 

    Filer kodes i øvrigt \emph{altid} i UTF-8-format.

    \section{main.c}
    \begin{itemize}
    \item Inkluderer \texttt{spi.h}.
    \item Benytter funktioner herfra.
    \end{itemize}

    \section{spi.h}
    \begin{itemize}
    \item Indeholder prototyper til globale funktioner.
    \item Indeholder tilhørende dokumentation af disse funktioner.
    \item Indeholder \texttt{typedef}s, \texttt{struct}s og \texttt{\#define}s tilhørende biblioteket. Generelt anvendes \texttt{struct}s uden \texttt{typedef}.
    \end{itemize}

    \section{spi.c}
    \begin{itemize}
    \item Inkluderer \texttt{spi.h}.
    \item Indeholder, øverst, statiske funktioner, der benyttes senere i globale funktioner.\footnote{\texttt{static}-funktioner benyttes for at dele koden om i simplere underfunktioner. ``End funktion skal gøre 1 ting, og gøre det godt!''. Bare tænk på $\sin(x)$.}
    \item Indeholder, nederst, kode til de globale funktioner, der er defineret i \texttt{spi.h}.
    \end{itemize}
\end{multicols}

\clearpage
\begin{landscape}
    \begin{multicols}{2}
        \chapter{Kodeeksempel}
        \section{spi.h}

        \begin{lstlisting}[language=c]
#include <stdio.h>

#ifndef SPI_H
#define SPI_H
#define RED 0x00FF0000
#define BLUE 0x0000FF00
#define GREEN 0x000000FF

#define MAX(A,B) a>b?a:b
#define udelay(us) _delay_us(us)

typedef unsigned char u8;

struct spi_adapter {
        unsigned char mosi;
        unsigned char miso;
        unsigned char sck;
};

/**
 * Sends one byte via SPI.
 * @param byte Data byte to be sent.
 * @param divide Number to divide the clock by.
 * @param slave ID of the slave to send to.
 *
 * @return 0 = success, SPI_ENACK = No acknowledge.
 */
int spi_send_byte(unsigned char byte, unsigned byte divide,
                  unsigned char slave);

#endif
        \end{lstlisting}

        \vfill
        \columnbreak
        \section{spi.c}
        \begin{lstlisting}[language=c]
#include "spi.h"

static void startclk()
{
        SPICTL |= 0x01;
}

int spi_send_byte(unsigned char byte, unsigned byte divide,
                  unsigned char slave)
{
        printf("Så er den sendt...\n");

        if (SPICTL & 0x01)
                return 0;
        else
                return SPI_ENACK;
}
        \end{lstlisting}

        \vfill
        \columnbreak

        \section{main.c}
        \begin{lstlisting}[language=c]
#include "spi.h"

int main(void)
{
        unsigned char input, c;

        input = getchar();
        switch (input) {
        case 0x01:
                spi_send_byte(0x01, 1, 0x01);
                break;
        case 0x02:
                c = spi_receive_byte();
                break;
        default:
                printf("Command unknown: %x\n", input);
                goto bailout;
        }

        /* 
         * Underneath are some different 
         * ways to use if-else statements
         * for this standard 
         */
    
        if (c > 100)
                printf("Det var et ordentligt tal!\n");
        else
                printf("Ja det fint...");

        if (c == 0x01) {  /* Switch colors. */
                change_color(BLUE);
                printf("Skiftet til blå");
        } else if (c == 0x02) {
                change_color(RED);
                printf("Skiftet til rød");
        } else {
                change_color(green);
                printf("Skiftet til grøn");
        }

        if (c < 10 || input == 0x03 || input == 0x04
            || input == 0x05) {
                print_help();
                delay(1000);
        }

        return 0;

bailout:
        return -1;
}
        \end{lstlisting}
    \end{multicols}
\end{landscape}

\end{document}
